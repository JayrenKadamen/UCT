% Chapter 5

\chapter{Conclusion} % Write in your own chapter title
\label{Chapter5}
\lhead{Chapter 5. \emph{Conclusion}} % Write in your own chapter title to set the page header

Objects in a scene were recognised using the height of their horizontal plane from the ground. The average height of each object type per room was then compared to a reference height. The discrepancy between the average and reference heights was expected to lie within $10\%$. This meant that a deviance of $7.4cm$ for desks and $4.9cm$ for chairs from their respective reference heights was expected. The maximum measured difference using normalised values was $3.77\%$ for desks which meant that desks were $2.8cm$ taller than their reference height (as detailed in chapter \ref{summaryresults}). The smallest difference between measured and reference heights were for chairs at $-0.30\%$ which resulted in them being $0.16cm$ shorter than their reference height. 

The objective of this thesis was accomplished. The scale was obtained using a fully-automatic method. The proof-of-concept test yielded results below the expected $10\%$ deviance from reference heights. The major find however was that the absolute measurement of objects yielded less accurate results compared to analysing the relationship between objects whose dimensions are based on human anatomy. 

The implication of this discovery has the potential to change the approach taken when using object recognition to automatically derive scale. Rather than detecting an object in a scene whose dimensions are known and using that to derive scale for the scene the relationship between different objects whose dimensions are based on human anatomy can be used to derive scale by using typical or expected values for their dimensions. 

The results obtained were from scenes on a university campus. There was variety between the scenes in terms of purpose and furniture; a computer lab (Geomatics Postgraduate Lab), a seminar room (EGS) and a modern classroom (Snape 3C). UCT does not have a policy that determines the dimensions for furniture but they do have a list of \href{http://www.uct.ac.za/usr/finance/pps/vendors/prefvend.pdf}{preferred vendors} which has been chosen based on having met quality standards. When researching furniture policies it became clear that the dimensions concerning furniture were not controlled to the exclusion of overall size so as to fit within an office or through the door for loading purposes. This allows one to rule out the possibility of obtaining misleadingly good results during the course of this research by having used scenes on a university campus as there exists no policy which determines furniture dimensions. In order to exhaustively test the proof-of-concept scenes in a different context must be used.

Other limits of the conclusions drawn from this research is that the typical or expected values for an object whose dimensions are based on human anatomy can vary between different countries. This is because human anatomy can vary between country populations, sometimes significantly enough to warrant differently scaled furniture for instance. In order to overcome this a greater variety of scenes must be tested. This includes spaces that are not situated on a university campus such as retail spaces, offices, and places of residence. 

