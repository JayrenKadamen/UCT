% Chapter 6

\chapter{Recommendations} % Write in your own chapter title
\label{Chapter6}
\lhead{Chapter 6. \emph{Recommendation}} % Write in your own chapter title to set the page header

This section will discuss possible avenues of further development in order to offer a full-realised solution.

\section{Object Recognition}
Object recognition with multiple descriptors can be used to improve the system rather than relying on recognising objects that fall within a particular height range. This will enable more accurate identification of objects within a greater variety of scenes. 

Implementing a fully-fledged object recognition system however will vary between platforms. For example desktops have more processing power and are often less restricted by internet usage and bandwidth concerns than mobile devices.  Possible implementations for these two types of platforms will be discussed in the subsections that follow.

\subsection{Development of a Smartphone Application}
Due to the widespread use of smartphones across the globe emphasis should be placed on developing an application for these devices that enable them to map their environment using the hardware currently available. This can be achieved by leveraging LSD-SLAM to generate the PCD and object recognition to obtain scale on the fly. LSD-SLAM can run in real-time and modern smartphones possess the processing power to perform this locally \citep{caruso_large-scale_2015}. The object recognition however can be performed on the device or sent to a server using on-board wireless technology such as WiFi or mobile networks. 

By performing the object recognition locally there needs to be a database of common objects (with a known dimension) one expects to find in any scene whether it be indoors or outdoors. This requires significant research for a number of reasons. A commonly found object may not have the same dimensions in every country. For example the average height of a chair in the United States may not be the same in Indonesia as the average height for the population may vary between each country as mentioned in chapter \ref{Chapter5}. Another concern is keeping the list of common objects short in order to minimise storage requirements for the application and to enable faster searches (as the database to search through will be smaller so the search will be faster). A fully-fledged database needs to be developed that is general enough to be used in any type of scene at any location which is efficient enough not to affect the real-time nature of the LSD-SLAM system. If all of these requirements were met then a fully-fledged mapping solution which can obtain a real-world scale could be performed directly on the device. This would be an extremely powerful tool without concerns such as cost for data transmission over mobile networks or the availability of free WiFi. 

If however the object recognition could be performed by a server then the local processing power can be dedicated to the LSD-SLAM system. Frames of the video feed can be sent to a server that is capable of CBIR such as \href{https://www.tineye.com/}{TinEye} so that objects within the image can be identified. These objects can then have dimensions stored locally or on a server.  The relevant scale can be sent to the device so that it can be applied to the PCD as it is being generated by the LSD-SLAM system. This allows the smartphone to access billions of images, far more than could be stored locally on the device. TinEye currently has an API that enables developers to create applications that make use of their CBIR service called \href{https://services.tineye.com/MobileEngine}{MobileEngine}. Using a server based approach relies on a wireless access point to be available which is not always the case. Multiple images will have to be sent which raises bandwidth concerns and possible data charges. The impact of using WiFi on battery life is also a factor that needs to be taken into account. 

Other areas not related to determining scale need to be assessed such as the memory requirements of such a system as massive amounts of data need to be captured in terms of video frames or multiple images and then converted to a point cloud representation. This affects both the RAM and local storage specifications. These problems have however been tackled by \href{https://www.google.com/atap/project-tango/}{Google's Project Tango} where a tablet variant is on offer which makes use of a 4 megapixel RGB-Infrared sensor that has access to 4Gb of RAM and a 128Gb SSD for rapid local read and write storage.  

\subsection{Web-based Applications}

There are billions of images on the internet which document places on Earth, many of which are well known. CBIR systems can sift through image databases such as flickr to find images of a particular scene such as a monument. SfM can then be used to combine images taken at different times with a variety of cameras. Object recognition that makes use of services such as TinEye can then identify objects within the images in order to derive a scale for the PCD. This system can be built into a web-based application that users can interact with through their web browser with all the processing being performed remotely on a server. The server could also have country-specific dimensions for common objects that are based on human anatomy for both indoor and outdoor spaces.  Users can also have the option of uploading their own images with the aim of recreating a 3D environment where they are the subject of the scene.

Both the private and government sectors can use this technology for a variety of tasks. Shopping malls can be digitally reconstructed and scaled using commonly found objects such as benches so that customers can access a 3D map and path finding system to find points of interest. Due to the size of a structure such as a mall this task would be more easily performed on a desktop once all the imagery had been uploaded to it. This concept can be extended to other areas, for example theme parks or museums such as the Louvre in Paris. Law enforcement can use this technology to derive real-world measurements such as heights of people from surveillance or public footage.

